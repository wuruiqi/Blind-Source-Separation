\begin{abstract}
Blind source separation (BSS) and independent component analysis (ICA) are generally based on a wide class of unsupervised learning algorithms and they found potential applications in many areas from engineering to neuroscience. A recent trend in BSS is to consider problems in the framework of matrix factorization or more general signals decomposition with probabilistic generative and tree structured graphical models and exploit a priori knowledge about true nature and structure of latent (hidden) variables or sources such as spatio-temporal decorrelation, statistical independence, sparseness, smoothness or lowest complexity in the sense e.g., of best predictability. The possible goal of such decomposition can be considered as the estimation of sources not necessary statistically independent and parameters of a mixing system or more generally as finding a new reduced or hierarchical and structured representation for the observed (sensor) data that can be interpreted as physically meaningful coding or blind source estimation. The key issue is to find a such transformation or coding (linear or nonlinear) which has true physical meaning and interpretation. We present a review of BSS and ICA, including various algorithms for static and dynamic models and their applications. The paper mainly consists of three parts: (1) BSS algorithms for static models (instantaneous mixtures); (2) extension of BSS and ICA incorporating with sparseness or non-negativity constraints; (3) BSS algorithms for dynamic models (convolutive mixtures).
\end{abstract}
\begin{keyword}
Independent Component Analysis \sep Blind Source Separation \sep information theory \sep feature extraction
\end{keyword}
